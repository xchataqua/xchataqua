\chapter{Schnellstart}
Wer sich mit dem IRC auskennt und schon die eigenheiten einiger IRC-Programme
kennengelernt hat, m�chte nicht unbedingt die ganze Dokumentation w�lzen um an
die jeweiligen Einstellungen des XChats zu kommen. Darum f�r alle jene, die
diese sch�ne Dokumentation missen m�chten hier ein Schnellstart ;)
Als Beispiel nehme ich den \texttt{irc.euirc.net} Server und als Kanal
\texttt{\#studies}.

\section{XChat f�r Linux/Unix}
Ich gehe davon aus, dass die Paketverwaltung der jeweiligen Distribution das
xchat Paket auf den Rechner gebracht hat. Sollte das nicht der Fall sein,
sollte man auf Seite \pageref{bekommen_starten} unter (\ref{bekommen_starten})
  vorbeischauen um den XChat zu installieren. 
  Gestartet wird der XChat durch das Kommando \verb|xchat-gnome|\footnote{In manchen Distributionen wie Debian
    heisst die ausf�hrbare Datei xchat-gnome. Abh�ngig ist dies vom Hersteller
    des Packetes f�r die Distribution.} oder allgemein \verb|xchat|.

\section{XChat f�r Windows}
Mit dem Browser geht man auf \texttt{http://www.xchat.org} und besorgt sich den
neusten XChat. Nach dem Download des Pakets, l�sst sich der XChat wie ein
normales Windowsprogramm �ber einen Installations-Wizard installieren.

\section{Verbindung aufbauen und chatten}
Zu Gesicht bekommt man das Serverfenster welches schon vorkonfiguriert einige
Netzwerke beinhaltet. Da es sich hier um einen Schnellstart handelt, soll uns diese
Liste nicht weiter interessieren, also wegklicken. Im Hauptfenster
verbindet man sich mit folgenden Kommandos zum Server:
\begin{quote}
  \texttt{/server irc.euirc.net}
\end{quote}

Nachdem absetzen des Kommandos sollte nach der Verbindung einiges an Text
in dem Textfenster zu sehen sein (Message of the Day, Verbindungs- und Benutzerstatistiken). Daraufhin kann 
man einen Kanal beitreten:
\begin{quote}
  \texttt{/j \#studies}
\end{quote}

Nebenbei kann man jetzt in aller Ruhe die Eigenheiten und Funktionen des XChats
kennenlernen, indem man diese Dokumentation liest.


