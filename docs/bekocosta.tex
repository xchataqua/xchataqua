
\chapter{Bekommen, Compilieren und Starten}\label{bekommen_starten}
\section{Was ist XChat?}

XChat ist ein grafischer IRC Client, welcher unter Unix �hnlichen Systemen l�uft. Es benutzt das GTK+ Toolkit f�r die grafische Oberfl�che. Es ist GPLed Software (Freie Software). 
Unter folgenden Systemen sollte es laufen:

\begin{itemize}
\item Linux (prim�re Entwicklungsplattform)
\item FreeBSD
\item OpenBSD
\item NetBSD
\item Solaris
\item AIX
\item IRIX
\item SunOS
\item OS/2
\item MS Windows
\end{itemize}


\section{Bekommen}
Wenn Du faul bist und nicht noch einige Hilfsbibliotheken installieren willst, hostet Peter Alexandrou eine \emph{X-Chat Paketseite}\footnote{http://www.users.bigpond.com/redowl/xchat/}.
Die Hauptdistribution gibt es von der XChat Homepage\footnote{http://xchat.linuxpower.org}. XChat ben�tigt GTK\footnote{http://www.gtk.org} und dazu kannst Du optional noch GNOME und PERL benutzen.

\section{Compilieren}
XChat benutzt das ``GNU autoconf system'', so dass das Compilieren sehr leicht sein sollte. F�r die meisten Systeme sollte die automatische Erkennung funktionieren:
\texttt{./configure ; make ; su ; make install}
Auf einigen Systemen wird gmake mehr gebraucht, als make. Dem Konfigurationsscript(configure) kann man noch einige Optionen �bergeben:
\begin{itemize}
\item --disable-perl = Schaltet die PERL Unterst�tzung aus
\item --disable-gnome = Schaltet die GNOME Unterst�tzung aus
\end{itemize}

Es sei darauf hingewiesen, dass das Script diese Optionen automatisch setzt, wenn kein GNOME oder PERL installiert ist. Sie sind nur f�r den Fall gedacht, wenn man GNOME oder PERL installiert hat, aber es nicht nutzen m�chte.
Sollte es bei dieser Methode Probleme geben, versuch die alte Methode:
\texttt{cp Makefile.gtk Makefile ; make ; su ; make install}

\section{Starten}

Die Compilierung erzeugt eine Bin�rdatei namens xchat. Wenn Du es installiert
hast, sollte XChat laufen, wenn man \verb|xchat| in die Konsole eintippt.
Ansonsten einfach in das XChat Verzeichnis und \verb|./xchat| eintippen.


Das Verzeichnis \verb|./xchat| sollte f�r Dich automatisch erstellt werden.
XChat benutzt das Verzeichnis, um benutzerspezifische Einstellungen und Logs
abzulegen.
